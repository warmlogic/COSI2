\documentclass[12pt,doc]{apa}
%\documentclass[12pt]{article}
%\usepackage{apacite} % \cite<prefix>[suffix]{key1,key2}, \citeA, \citeNP,
                     % \fullcite, \shortcite, \citeyear, \citeauthor, \nocite
\synctex=1
%\usepackage[margin=1.5in]{geometry}
%\usepackage[left=1in,top=1in,right=1in,bottom=1in,nohead]{geometry}
%\geometry{letterpaper} 
%\geometry{landscape} % Rotated page geometry
%\usepackage{color}
\usepackage{soul} % \hl \ul \st \caps \so
\usepackage{graphicx}
\usepackage{amsmath}
%\usepackage{epstopdf}
%\usepackage{setspace}
%\doublespacing
%\setstretch{1.66}

\usepackage{fancyhdr}
\setlength{\headheight}{15.2pt}
% no horizontal rule
\renewcommand{\headrulewidth}{0pt}
\pagestyle{fancy}

\fancyhead{} % delete current setting for header
\fancyhead[R]{Experiment: COSI2}

% \title{RA Education:\\COSI2: Memory for contextual information}
% \author{~}
% \date{~}%\today
% \affiliation{~}

\begin{document}
%\maketitle
%\section{}
%\subsection{}

\begin{center}
\large{RA Education:}

\large{COSI2: Memory for contextual information}
\end{center}

\vspace{0.1cm}

\section*{Background}

In the dual-process framework of recognition memory, familiarity and
recollection are the two main cognitive processes involved in
remembering information.  Familiarity is typically thought to involve
a fast and automatic recognition process that allows for an awareness
of a previous experience with an item without retrieval of details
from the encoding episode, whereas recollection is a slower process
that retrieves item-specific episodic information \cite{Yone2002}

In this study, we investigated the contribution of familiarity and
recollection to source memory.  Source memory involves remembering
contextual details, such as having a memory for the person from whom
you heard a juicy rumor, or discriminating between whether you said
something out loud or just thought it internally (the ``source'' of
the information).  The retrieval of episodic information has sometimes
been defined as a property of the recollection process
\shortcite{RuggEtal1998b,Tulv1985a,Yone2002}, and accurate source
recognition has been considered a defining feature of recollection.
However, computational models of familiarity-based recognition have
been shown to be capable of supporting source recognition
\shortcite{ElfmEtal2008}, and a variety of empirical evidence reviewed
next has also suggested that familiarity contributes to source
recognition under some conditions.  These familiarity effects have
been indexed behaviorally
\shortcite<e.g.,>{ElfmEtal2008,HickEtal2002}, by the FN400
event-related potential (ERP) component
\shortcite<e.g.,>{EckeEtal2007a,EckeEtal2007b}, by activity in the
perirhinal cortex \shortcite<which is thought to be related to
familiarity; e.g.,>{DianEtal2007}, and in neuropsychological patients
with hippocampal damage thought to impair recollection
\shortcite<e.g.,>{QuamEtal2007}.


A number of experiments have used the Remember--Know (RK) procedure to
assess the correlates of recollection and familiarity to source
memory.  Here, ``remember'' and ``know'' responses are thought to be
subjective indices of recollection and familiarity, respectively
\shortcite<e.g.,>{RuggEtal1998b,Tulv1985a}, and some studies have suggested
that familiarity can contribute to source information.  For example,
\shortciteA{HickEtal2002} used two experiments to investigate familiarity's
contribution to source monitoring, and found source accuracy for
``know'' responses to be equal to that of ``remember'' responses,
which suggests that a sense of familiarity is sufficient to contribute
to successful source monitoring.  As an alternative to subjective
behavioral reports, recollection and familiarity have been associated
with particular ERP effects \shortcite<for reviews,
see>{Curr2000,RuggCurr2007}).  The {\it parietal ERP old/new effect}
is thought to reflect processing related to recollection, while
familiarity is thought to be indexed by the {\it frontal old/new
  effect} or the {\it FN400}.

Focusing more on the properties of encoded stimuli, which are
important in a source memory experiment, {\it intrinsic} source
features are intra-item features, meaning they are part of the
perceived stimulus, such as the paint color of a car, and {\it
  extrinsic} source features are external (or inter-item)
associations, such as the context in which you saw a particular car.
Ecker and colleagues found a FN400 source accuracy effect for the
intrinsic but not the extrinsic case, meaning familiarity was
sensitive to intrinsic features
\shortcite{EckeEtal2007a,EckeEtal2007b}.  However, other studies have
suggested that familiarity can contribute to memory for extrinsic
associations or sources.  \shortciteA{SpeeCurr2007} demonstrated that
the FN400 could correctly differentiate between old and new extrinsic
associations between fractal images.

\section*{Our experiment}

The aim of the present experiment was to test whether familiarity, as
indexed by both behavioral measures and the FN400 mid-frontal old/new
effect, is able to differentiate certain kinds of extrinsic perceptual
features encoded as source information.  In one modality, spatial
location was used as the source detail to investigate the potential
role of familiarity in recognizing an extrinsic attribute.  In the
other modality, extrinsic color associations were used as the source
details.  Participants studied pictures of objects on 8 lists (100
objects on each list; 4 lists in session 1, 4 in session 2) that were
paired with either spatial source (left or right side of the screen)
or color source (a blue or yellow frame around the object).  After
each study list was a test period where we used a modified RK
procedure to probe recollection and familiarity in a subjective
manner, while also recording EEG to investigate source memory more
objectively with ERPs.  During test we randomly intermixed the 100
studied (old) objects with 50 new objects, each presented without its
source.  Participants first had to tell us either the source
(left/right or blue/yellow) or whether it was a new object.  If they
answered with a source, they did the RK procedure where they told us
whether they specifically remembered source information for that
object, whether they remembered something else, or whether the object
just felt familiar.

Overall, both the ERP and behavioral results suggest that
familiarity's contribution to extrinsic source monitoring depends on
the type of source information being remembered.  Specifically,
familiarity seems to contribute to remembering spatial source
information while it does not help with remembering extrinsic color
associations.  There was an FN400 ERP effect for old objects in the
spatial source condition when getting the source correct compared to
getting it incorrect.  Additionally, the ``Familiar'' RK response
accuracy was above chance only for spatial source information.
Neither of these effects were seen in the color source condition.

\section*{More information}

We have submitted a paper on this experiment for publication, but it
has not yet been accepted.  If you would like more information, please
contact Matt Mollison at matthew.mollison@colorado.edu, and/or check
out the references below.


\bibliography{mvm}
%\bibliographystyle{apacite}

\end{document}
